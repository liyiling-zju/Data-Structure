
\documentclass[twoside,a4paper]{article}
\usepackage{geometry}
\geometry{margin=1.5cm, vmargin={0pt,1cm}}
\setlength{\topmargin}{-1cm}
\setlength{\paperheight}{29.7cm}
\setlength{\textheight}{25.3cm}

% useful packages.
\usepackage{amsfonts}
\usepackage{amsmath}
\usepackage{amssymb}
\usepackage{amsthm}
\usepackage{enumerate}
\usepackage{graphicx}
\usepackage{multicol}
\usepackage{fancyhdr}
\usepackage{layout}
\usepackage{CJKutf8}
\usepackage{listings}

% some common command
\newcommand{\dif}{\mathrm{d}}
\newcommand{\avg}[1]{\left\langle #1 \right\rangle}
\newcommand{\difFrac}[2]{\frac{\dif #1}{\dif #2}}
\newcommand{\pdfFrac}[2]{\frac{\partial #1}{\partial #2}}
\newcommand{\OFL}{\mathrm{OFL}}
\newcommand{\UFL}{\mathrm{UFL}}
\newcommand{\fl}{\mathrm{fl}}
\newcommand{\op}{\odot}
\newcommand{\Eabs}{E_{\mathrm{abs}}}
\newcommand{\Erel}{E_{\mathrm{rel}}}

\begin{document}
\begin{CJK*}{UTF8}{gkai}

\pagestyle{fancy}
\fancyhead{}
\lhead{22035037 数学科学学院 计算数学专业}
\chead{数据结构作业 \#05}
\rhead{2020/11/23}


\section*{ 二叉搜索树的模板化实现}

\subsection*{a 文件说明}
在$BinarySearchTree.h$文件中定义一个类,实现二叉搜索树的min、max、successor、predecessor、tree\_search、insert、

\noindent delete、inorder\_walk等功能,在$main.cpp$文件中测试各项功能,类构造及测试设计如下:

BinarySearchTree$\left< \rm{T} \right> $

    |---class--Node:包含一个T类型关键字和三个分别指向left kid、right kid和parent的指针;
    
    |---function--insert(T):向二叉树中插入关键字为给定值的结点;
    
    |---function--min(Node*):给定一个结点,返回一个指向以该结点为根的子树中最小元素的指针;
    
    |---function--max(Node*):给定一个结点,返回一个指向以该结点为根的子树中最大元素的指针;
    
    |---function--successor(Node*):给定一个结点,返回一个指向以该结点的后继的指针;
    
    |---function--predecessor(Node*):给定一个结点,返回一个指向以该结点的前驱的指针;

    |---function--tree\_delete(Node*):从二叉树中删除给定结点;
    
    |---function--tree\_search(Node*, T):查找关键字等于给定值的结点并返回指向给结点的指针;

    |---function--inorder\_walk():从根节点开始,升序输出二叉树中结点对应的关键字;

    |---function--RandArray(T*,int):将输入的数组打乱顺序;
    
    |---function--test():测试successor(Node*),predecessor(Node*),min(Node*),max(Node*),tree\_search(Node*,T)及

    tree\_delete(Node*)函数的测试;

\subsection*{b 测试方案及结果}

\noindent 1.测试方案

\noindent (1)对类中各函数功能的测试
\begin{lstlisting}[language=R]
template <typename T>
void BinaryTree<T>::test()
{
  std::cout << "min(root)=" << min(root)->data << std::endl;
  std::cout << "max(root)=" << max(root)->data << std::endl;
  std::cout << "successor(root):" << successor(root)->data << std::endl;
  std::cout << "predecessor(root):" << predecessor(root)->data << std::endl;
  std::cout << "tree\_search(root,12):" << tree\_search(root,(T)12) << std::endl;
  tree_delete(root->lc);
  std::cout << "after tree\_delete(root->lc): " << std::endl;
  inorder_walk();
};
\end{lstlisting}

\noindent (2)排序:测试乱序函数、插入及中序输出

\noindent · 打乱一个数组中的元素
\begin{lstlisting}[language=R]
template <typename T>
void BinaryTree<T>::RandArray(T* A, int n)
{
  T a;
  int x;
  srand(time(NULL));
  for(int i = 0; i < n; i++)
    {
      x=rand() % n;
      a = A[i];
      A[i] = A[x];
      A[x] = a;
    }
};
\end{lstlisting}

\noindent · 利用insert函数将该数组插入,再利用inorder\_walk函数中序输出;


\noindent 2. 测试结果

\noindent 下面是一次测试的结果(RandArray函数每次产生的随机序列都不同,随机后的数组也会有所不同)。

\noindent To sort :B = [ 11 12 13 14 15 16 17 18 19 20 ]

\noindent random B = [ 14 15 18 16 19 20 13 17 11 12 ]

\noindent result of BST\_Sort:

\noindent 11	12	13	14	15	16	17	18	19	20

\noindent min(root) = 11

\noindent max(root) = 20

\noindent successor(root): 15

\noindent predecessor(root): 13

\noindent tree\_search(root,12): 0x561393dd9470

\noindent after tree\_delete(root-$>$lc):

\noindent 11	12	14	15	16	17	18	19	20



  
\subsection*{c  排序稳定性和效率分析}

\noindent 1. 稳定性

针对原始的BST算法,insert函数将小于作为left kid,大于等于作为right kid,结合inorder\_walk函数定义,一定是先遍历升序输出左子树,接着输出根节点,再遍历输出右子树,因此关键字相同的结点按照输入顺序输出,故基于BST的排序算法是稳定的;

针对先乱序再排序的BST排序算法,由于乱序会导致不稳定,因此最终结果也是不稳定的;

\noindent 2. 效率

分析两种算法的实现,根据RandArray函数定义,将数组随机打乱是$O(n)$的,两种算法除随机部分及比较的顺序不同之外其余步骤均相同,随机快排也可以写为二叉搜索树的形式,因此两种算法效率差异不大,平均运行时间均为$\Theta (n\lg n)$。


\end{CJK*}
\end{document}

%%% Local Variables: 
%%% mode: latex
%%% TeX-master: t
%%% End: 
